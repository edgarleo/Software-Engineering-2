\documentclass{article}
\usepackage{graphicx}
\usepackage[fontsize=11pt]{scrextend}
\usepackage{hyperref}
\title{Code Inspection}
\begin{document}

\begin{titlepage}
\begin{figure}
	\centering
	\includegraphics{polimi}
\end{figure}
\maketitle
\centering
Prof. Luca Mottola
\newline
\raggedleft
Authors:
\begin{itemize}
	\raggedleft
	\item ZHOU YINAN(Mat. 872686)
	\item ZHAO KAIXIN(Mat. 875464)
	\item ZHAN YUAN(Mat. 806508)	
\end{itemize}
\end{titlepage}

\tableofcontents
\newpage

\section{Classes Assigned}
We have been assigned two classes :
\begin{itemize}
	\item ProductionRun.java 
	\item ViewerServletRequest.java
\end{itemize}
The namespace patten is :
\begin{table}[h]
	\centering
	\label{my-label}
	\begin{tabular}{|l|}
		\hline
		../apache-ofbiz-16.11.01/applications/manufacturing/src/main/java/org/apache/ofbiz\\/manufacturing/jobshopmgt/ProductionRun.java \\ \hline
	\end{tabular}
\end{table}

\begin{table}[h]
	\centering

	\label{my-label}
	\begin{tabular}{|l|}
		\hline
		../apache-ofbiz-16.11.01/specialpurpose/birt/src/main/java/org/apache/ofbiz/birt\\/report/servlet/ViewerServletRequest.java \\ \hline
	\end{tabular}
\end{table}
\newpage
\section{Functional Role}
 \subsection{ProductionRun.java}
 Instead of directly looking into the code, we first examine the online ofbiz document to get information of this class. This class belongs to \textbf{Manufacturing} section. 
 The link to the 
 \href{https://cwiki.apache.org/confluence/display/OFBIZ/Beginner%27s+Guide+to+the+Manufacturing+Process}{Online Documet}. \\
\\The situation is described here. After a client makes order, configurable goods which our company provide require some type of manufacturing or production. If we do not have the requiring parts in our inventory, a production run is generated.

\begin{figure}[h]
	\centering
	\includegraphics[width=\textwidth]{web}
\end{figure}
If we log into the ofbiz web application, we can examine the production run section.
\begin{figure}[h]
	\centering
	\includegraphics[width=\textwidth]{productionRun}
\end{figure}
The ProductionRun class manages all the information of a certain production run activity. 
\newpage
 \subsection{ViewerServeltRequest}
 By looking at the code, we find that ViewerservletRequest extends HttpServletRequestWrapper. A "HttpServletRequestWrapper" provides a convenient implementation of the HttpServletRequest interface that can be subclassed by developers wishing to adapt the request to a Servlet. Thus the  role of this class is to represent a specific function of HttpServletRequest. More specifically, this function is getParameter();\\
 \\Before looking into this function, let's recall what is a servlet. A servlet lives in a web container, and it is responsible for generating dynamic web contents. Servlet can be viewed as a special java class without main methods. After a client sends a HTTP request to the web server, the web container is responsible for :
 \begin{itemize}
 	\item create an instance of a servlet
 	\item call specific method of a servlet
 	\item destroy a servlet
 \end{itemize}
\begin{figure}[h]
	\centering
	\includegraphics[width=\textwidth]{web_container}
\end{figure}

\begin{figure}[h]
\centering
\includegraphics[width=\textwidth]{handle_request}
\end{figure}
\newpage
The web container knows which servlet to call because a servlet can have three names :
\begin{itemize}
	\item Client knows URL name
	\item Deployer knows servlet secret internal name
	\item Actual java class name
\end{itemize}
 The XML document is responsible for deployment.
 \begin{figure}[h]
 	\centering
 	\includegraphics[scale = 0.7]{xml}
 \end{figure}
\newpage
Now let's look at what function role is this class. \textbf{ServletRequest} defines an object to provide client request information to a servlet. The servlet container creates a ServletRequest object and passes it as an argument to the servlet's service method. A \textbf{ServletRequest} object provides data including parameter name and values, attributes, and an input stream.\\
This java class file is used to form the parameter.
\section{Check List}
 \subsection{ProductionRun.java}
  \subsubsection{Naming Conventions}
  \subsubsection{Indention}
  \subsubsection{Braces}
  \subsubsection{File Organization}
  \subsubsection{Wrapping Lines}
  \subsubsection{Comments}
  \subsubsection{Java Source Files}
  \subsubsection{Package and Import Statements}
  \subsubsection{Class ans Interface Declarations}
  \subsubsection{Initialization and Declarations}
  \subsubsection{Method Calls}
  \subsubsection{Arrays}
  \subsubsection{Object Comparison}
  \subsubsection{Output Format}
  \subsubsection{Computation, Comparisons, and Assignments}
  \subsubsection{Exceptions}
  \subsubsection{Flow of Control}
  \subsubsection{Files}

\newpage
 \subsection{ViewerServletRequest}
 \subsubsection{Naming Conventions}
 \subsubsection{Indention}
 \subsubsection{Braces}
 \subsubsection{File Organization}
 \subsubsection{Wrapping Lines}
 \subsubsection{Comments}
 \subsubsection{Java Source Files}
 \subsubsection{Package and Import Statements}
 \subsubsection{Class ans Interface Declarations}
 \subsubsection{Initialization and Declarations}
 \subsubsection{Method Calls}
 \subsubsection{Arrays}
 \subsubsection{Object Comparison}
 \subsubsection{Output Format}
 \subsubsection{Computation, Comparisons, and Assignments}
 \subsubsection{Exceptions}
 1. The relevant exception is caught. 
 \begin{verbatim}
 try {
 reportFileUrl = FlexibleLocation.resolveLocation(reportParam, loader);
 } catch (MalformedURLException e) {
 Debug.logError(e, module);
 }
 if (reportFileUrl == null) {
 throw new IllegalArgumentException("Could not resolve location to URL: " 
 + reportParam);
 }
 \end{verbatim}
 The function in this code block tries to locate the file url. In case of wrong url and no file found, an exception is raised.
 \subsubsection{Flow of Control}
 No \textbf{switch} and \textbf{loop} in this file.
 \subsubsection{Files}
 This java class does not deal with file operations. 
\end{document}